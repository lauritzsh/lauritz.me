\documentclass[10pt,a4paper,ragged2e,dvipsnames]{altacv}
\geometry{left=2cm,right=10cm,marginparwidth=6.8cm,marginparsep=1.2cm,top=1.25cm,bottom=1.25cm}
\ifxetexorluatex
\usepackage{fontspec}
\renewcommand{\familydefault}{\sfdefault}
% \setmainfont{Helvet}
\else
\usepackage[utf8]{inputenc}
\usepackage[T1]{fontenc}
\usepackage[default]{lato}
\fi
\definecolor{VividPurple}{HTML}{000000}
\definecolor{SlateGrey}{HTML}{2E2E2E}
\definecolor{LightGrey}{HTML}{2E2E2E}
\colorlet{heading}{VividPurple}
\colorlet{accent}{VividPurple}
\colorlet{emphasis}{SlateGrey}
\colorlet{body}{LightGrey}
\renewcommand{\itemmarker}{{\small\textbullet}}
  \renewcommand{\ratingmarker}{\faCircle}
  \addbibresource{sample.bib}

  \begin{document}
  \name{Lauritz Hilsøe}
  \tagline{Softwareudvikler}
  % Cropped to square from https://en.wikipedia.org/wiki/Marissa_Mayer#/media/File:Marissa_Mayer_May_2014_(cropped).jpg, CC-BY 2.0
  %\photo{3.3cm}{profile.jpg}
  \personalinfo{%
    % Not all of these are required!
    % You can add your own with \printinfo{symbol}{detail}
    \email{mail@lauritz.me}
    \phone{+45 53 61 67 69}
    %  \mailaddress{Address, Street, 00000 County}
    \location{København N, Danmark}
    %  \homepage{marissamayr.tumblr.com/}
    %  \twitter{@marissamayer}
    \\
    \github{https://github.com/lauritzsh/}
    \linkedin{https://www.linkedin.com/in/lauritzhilsoe/}
    %   \orcid{orcid.org/0000-0000-0000-0000} % Obviously making this up too. If you want to use this field (and also other academicons symbols), add "academicons" option to \documentclass{altacv}
    }

    %% Make the header extend all the way to the right, if you want.
    \begin{fullwidth}
      \makecvheader
    \end{fullwidth}

    %% Depending on your tastes, you may want to make fonts of itemize environments slightly smaller
    \AtBeginEnvironment{itemize}{\small}

    %% Provide the file name containing the sidebar contents as an optional parameter to \cvsection.
    %% You can always just use \marginpar{...} if you do
    %% not need to align the top of the contents to any
    %% \cvsection title in the "main" bar.
    \cvsection[page1sidebar]{Erfaring}

    \cvevent{Softwareudvikler}{Shape A/S}{Jan. 2019 -- Jun. 2020}{København S}
    \begin{itemize}
      \item Stod for research til opstart af projekt og løbende udvikling af projekt.
      \item Fik lært om udvikling af større projekter fordelt over flere hold samt kommunikation mellem
        disse hold til diverse møder.
      \item Fik afprøvet flere roller som både frontend- og backend-udvikler.
      \item På grund af corona fik jeg også erfaring hvordan man arbejder remote med folk.
    \end{itemize}

    \divider

    \cvevent{Studenter softwareudvikler}{2BM A/S}{Nov. 2017 -- Sep. 2018}{København Ø}
    \begin{itemize}
      \item Arbejde med SAPUI5 og iOS/Swift.
    \end{itemize}

    \divider

    \cvevent{Teaching Assistent}{ITU}{Jan. 2018 -- Maj 2018}{København S}
    \begin{itemize}
      \item Hjælpelærer for kurset ``digitalt materiale og interaktive artefakter''.
      \item Lærte at skulle formidle ideeer og koncepter man tager forgivet til folk der for første gang skal lære om det.
      \item At forberede sig og holde forelæsning foran et hold.
      \item At skulle rette og vurderes de studerendes præstationer i opgaveafleveringer.
    \end{itemize}

    \divider

    \cvevent{Studenter softwareudvikler}{WAYF}{Sep. 2015 -- Aug. 2016}{København S}
    \begin{itemize}
      \item Udvikling af interne systemer
    \end{itemize}

    \cvsection{Færdigheder}
    \smallskip
    \begin{tabular}{rl}
      \makeatletter
      \textsc{\textbf{Programmeringssprog}} & TypeScript, JavaScript \\
      & Python, SQL, Elixir, Kotlin \\
      \textsc{\textbf{Softwareudvikling}} & Git, Testing, CLI, Agile \\
      \textsc{\textbf{Frontend}} & React, Redux, Vue \\
      \textsc{\textbf{Backend}} & Node.js, Next.js, Phoenix, Ktor \\
      \textsc{\textbf{Databaser}} & Postgres \\
      \textsc{\textbf{Sprog}} & Dansk, Engelsk \\
      \makeatother
    \end{tabular}

    \cvsection{Interesser}
    \smallskip
    \begin{itemize}
      \item Webudvikling, funktionel programmering og machine learning
      \item Elektronisk musik
      \item Badminton og styrketræning
      \item Lære ting (sprog, teknologier, koncepter, ideer)
      \item Bo i udlandet (over tre måneder)
      \item Finans og markeder (aktier, optioner, krypto)
    \end{itemize}

    % \cvevent{Product Engineer}{Google}{23 June 1999 -- 2001}{Palo Alto, CA}

    % \begin{itemize}
    % \item Joined the company as employe \#20 and female employee \#1
    % \item Developed targeted advertisement in order to use user's search queries and show them related ads
    % \end{itemize}

    %\cvsection{A Day of My Life}

    % Adapted from @Jake's answer from http://tex.stackexchange.com/a/82729/226
    % \wheelchart{outer radius}{inner radius}{
    % comma-separated list of value/text width/color/detail}
    % Some ad-hoc tweaking to adjust the labels so that they don't overlap
    % \wheelchart{1.5cm}{0.5cm}{%
    %   10/10em/accent!30/Sleeping \& dreaming about work,
    %   25/9em/accent!60/Public resolving issues with Yahoo!\ investors,
    %   5/13em/accent!10/\footnotesize\\[1ex]New York \& San Francisco Ballet Jawbone board member,
    %   20/15em/accent!40/Spending time with family,
    %   5/8em/accent!20/\footnotesize Business development for Yahoo!\ after the Verizon acquisition,
    %   30/9em/accent/Showing Yahoo!\ employees that their work has meaning,
    %   5/8em/accent!20/Baking cupcakes
    % }

    \clearpage

    % \cvsection[page2sidebar]{Publications}

    \nocite{*}

    % \printbibliography[heading=pubtype,title={\printinfo{\faBook}{Books}},type=book]

    % \divider

    % \printbibliography[heading=pubtype,title={\printinfo{\faFileTextO}{Journal Articles}}, type=article]

    % \divider

    % \printbibliography[heading=pubtype,title={\printinfo{\faGroup}{Conference Proceedings}},type=inproceedings]

    % %% If the NEXT page doesn't start with a \cvsection but you'd
    % %% still like to add a sidebar, then use this command on THIS
    % %% page to add it. The optional argument lets you pull up the
    % %% sidebar a bit so that it looks aligned with the top of the
    % %% main column.
    % % \addnextpagesidebar[-1ex]{page3sidebar}


  \end{document}
