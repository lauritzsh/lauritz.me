\cvsection{Uddannelse}
\cvedu{Bachelor -- ML og Datavidenskab}{Københavns Universitet}{2020 -- nu}{København Ø, Danmark}
\cvedu{Kandidat -- Softwareudvikling}{IT-Universitetet i København}{2016 -- 2018}{København S, Danmark}
\cvedu{Udveksling}{Auckland University of Technology}{2017}{Auckland, New Zealand}
\cvedu{Bachelor -- Softwareudvikling}{IT-Universitetet i København}{2013 -- 2016}{København S, Danmark}
% \cvachievement{\faTrophy}{}{Received accolades at Atos for Best Performance in team.}
% \cvachievement{\faTrophy}{}{Received Best Debut Award at Atos. }
% %\divider
% \cvachievement{\faInstitution}{}{Won 2nd Consolation Prize for paper presented on Cognitive Radio Networks.}
% %\divider
% \cvachievement{\faGraduationCap}{}{Got Selected in "Exclusive Scholar Program" during undergrad.}
% %\divider
% \cvachievement{\faDollar}{}{Awarded with Narotam Sekhsaria Foundation Scholarship}
%\cvsection{Strengths}

%\cvtag{Hard-working (18/24)} 
%\cvtag{Persuasive}
%\cvtag{Motivator \& Leader}

%\divider\smallskip

%\cvtag{UX}
%\cvtag{Mobile Devices \& Applications}
%\cvtag{Product Management \& Marketing}


%\divider

%\cvevent{B.S.\ in Symbolic Systems}{Stanford University}{Sept 1993 -- June 1997}{}

\cvsection{Projekter}
\cvproject{Cryptomon}
\begin{itemize}
  \item Er interesseret i både webudvikling og funktionelle sprog. Dette
    projekt var et forsøg på at kombinere dem ved at benytte Reason og
    ReasonReact. Projektet endte med at blive en simpel `crypto currency monitor' som
    kan holde øje med en brugers portefølje og hvor godt det går.
\end{itemize}
\smallskip
\smallskip
\cvproject{ShiftPlanning web}
\begin{itemize}
  \item På ITU er der en studenterdreven kaffebar. De havde brug for et
    planlægningssystem til vagter, og sammen med nogle venner udviklede vi
    sådan et system. Jeg var ansvarlig for deres web app.
  \item Projektet lærte mig om struktur og kompleksitet og at bruge de rigtige
    værktøjer. Redux var unødvendigt til dette, men som det nye hotte der,
    måtte jeg også bare bruge det. I dag havde jeg gjort det anderledes og
    holdt mig mere til ren React.
\end{itemize}
\smallskip
\smallskip
\cvproject{fourchan-kit}
\begin{itemize}
  \item Mit første end-to-end projekt, hvor jeg lavede en Ruby gem, som et Ruby
    API til 4chan hjemmesiden. Med den kan brugeren læse indslag, hente
    billeder og mere.
  \item Her fik jeg lært en del, især om dokumentation og testing, og hvad det
    vil sige at skrive et godt API.
\end{itemize}
