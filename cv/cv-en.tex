\documentclass[10pt,a4paper,ragged2e,dvipsnames]{altacv}
\geometry{left=2cm,right=10cm,marginparwidth=6.8cm,marginparsep=1.2cm,top=1.25cm,bottom=1.25cm}
\ifxetexorluatex
\usepackage{fontspec}
\renewcommand{\familydefault}{\sfdefault}
% \setmainfont{Helvet}
\else
\usepackage[utf8]{inputenc}
\usepackage[T1]{fontenc}
\usepackage[default]{lato}
\fi
\usepackage{hyperref}
\definecolor{VividPurple}{HTML}{000000}
\definecolor{SlateGrey}{HTML}{2E2E2E}
\definecolor{LightGrey}{HTML}{2E2E2E}
\colorlet{heading}{VividPurple}
\colorlet{accent}{VividPurple}
\colorlet{emphasis}{SlateGrey}
\colorlet{body}{LightGrey}
\renewcommand{\itemmarker}{{\small\textbullet}}
  \renewcommand{\ratingmarker}{\faCircle}
  \addbibresource{sample.bib}

  \begin{document}
  \name{Lauritz Hilsøe}
  \tagline{Software Developer}
  % Cropped to square from https://en.wikipedia.org/wiki/Marissa_Mayer#/media/File:Marissa_Mayer_May_2014_(cropped).jpg, CC-BY 2.0
  %\photo{3.3cm}{profile.jpg}
  \personalinfo{%
    % Not all of these are required!
    % You can add your own with \printinfo{symbol}{detail}
    \email{mail@lauritz.me}
    \phone{+45 53 61 67 69}
    %  \mailaddress{Address, Street, 00000 County}
    \location{Frederiksberg C, Danmark}
    %  \homepage{marissamayr.tumblr.com/}
    %  \twitter{@marissamayer}
    \\
    \github{https://github.com/lauritzsh/}
    \linkedin{https://www.linkedin.com/in/lauritzhilsoe/}
    %   \orcid{orcid.org/0000-0000-0000-0000} % Obviously making this up too. If you want to use this field (and also other academicons symbols), add "academicons" option to \documentclass{altacv}
    }

    %% Make the header extend all the way to the right, if you want.
    \begin{fullwidth}
      \makecvheader
    \end{fullwidth}

    %% Depending on your tastes, you may want to make fonts of itemize environments slightly smaller
    \AtBeginEnvironment{itemize}{\small}

    %% Provide the file name containing the sidebar contents as an optional parameter to \cvsection.
    %% You can always just use \marginpar{...} if you do
    %% not need to align the top of the contents to any
    %% \cvsection title in the "main" bar.
    \cvsection[cv-en-sidebar]{Experience}

    \cvevent{Senior frontend developer}{NORD.investments A/S}{Sep. 2022 -- Dec. 2023}{Copenhagen, Denmark}
    \begin{itemize}
      \item Successfully migrated all existing projects from JavaScript to TypeScript, enhancing code readability and maintainability.
      \item Transitioned the codebase from an intricate polyrepo setup to a modern monorepo setup.
      \item Entrusted with initiating and developing a large-scale, multifaceted project.
      \item Improved the CI/CD pipeline, contributing to better workflow efficiency.
    \end{itemize}

    \divider

    \cvevent{Software developer}{Shape A/S}{Jan. 2019 -- Jun. 2020}{Copenhagen, Denmark}
    \begin{itemize}
      \item Responsible for research of technology for new web app and continuous development of it.
      \item Learned about development of larger projects spread across multiple
      teams and communication between these teams.
      \item Tried roles as both frontend and backend developer.
      \item Because of COVID-19 I got experience how to work remotely.
    \end{itemize}

    \divider

    \cvevent{Student software developer}{2BM A/S}{Nov. 2017 -- Sep. 2018}{Copenhagen, Denmark}
    \begin{itemize}
      \item Worked with SAPUI5 and Swift to build a mobile application.
    \end{itemize}

    \divider

    \cvevent{Teaching Assistent}{ITU}{Jan. 2018 -- May 2018}{Copenhagen, Denmark}
    \begin{itemize}
      \item TA for the course ``digital materials and interactive artefacts''.
      \item Learned how to convey ideas and concepts one takes for granted to people who are new to them.
      \item To prepare and give a lecture for a class.
      \item To evaluate the students homework and provide them feedback.
    \end{itemize}

    \divider

    \cvevent{Student software developer}{WAYF}{Sep. 2015 -- Aug. 2016}{Copenhagen, Denmark}
    \begin{itemize}
      \item Development of internal systems.
    \end{itemize}

    % \cvevent{Product Engineer}{Google}{23 June 1999 -- 2001}{Palo Alto, CA}

    % \begin{itemize}
    % \item Joined the company as employe \#20 and female employee \#1
    % \item Developed targeted advertisement in order to use user's search queries and show them related ads
    % \end{itemize}

    %\cvsection{A Day of My Life}

    % Adapted from @Jake's answer from http://tex.stackexchange.com/a/82729/226
    % \wheelchart{outer radius}{inner radius}{
    % comma-separated list of value/text width/color/detail}
    % Some ad-hoc tweaking to adjust the labels so that they don't overlap
    % \wheelchart{1.5cm}{0.5cm}{%
    %   10/10em/accent!30/Sleeping \& dreaming about work,
    %   25/9em/accent!60/Public resolving issues with Yahoo!\ investors,
    %   5/13em/accent!10/\footnotesize\\[1ex]New York \& San Francisco Ballet Jawbone board member,
    %   20/15em/accent!40/Spending time with family,
    %   5/8em/accent!20/\footnotesize Business development for Yahoo!\ after the Verizon acquisition,
    %   30/9em/accent/Showing Yahoo!\ employees that their work has meaning,
    %   5/8em/accent!20/Baking cupcakes
    % }

\cvsection{Skills}
    \smallskip
    \begin{tabular}{rl}
      \makeatletter
      \textsc{\textbf{Coding}} & TypeScript, JavaScript \\
      & Python, SQL, Kotlin, Elixir \\
      \textsc{\textbf{Frontend}} & React \& Redux, Remix, Next.js, Vue \\
      \textsc{\textbf{Backend}} & Ktor, Phoenix, Postgres, SQLite \\
      \textsc{\textbf{Languages}} & Danish, English \\
      \makeatother
    \end{tabular}







    \clearpage

    % \cvsection[page2sidebar]{Publications}

    \nocite{*}

    % \printbibliography[heading=pubtype,title={\printinfo{\faBook}{Books}},type=book]

    % \divider

    % \printbibliography[heading=pubtype,title={\printinfo{\faFileTextO}{Journal Articles}}, type=article]

    % \divider

    % \printbibliography[heading=pubtype,title={\printinfo{\faGroup}{Conference Proceedings}},type=inproceedings]

    % %% If the NEXT page doesn't start with a \cvsection but you'd
    % %% still like to add a sidebar, then use this command on THIS
    % %% page to add it. The optional argument lets you pull up the
    % %% sidebar a bit so that it looks aligned with the top of the
    % %% main column.
    % % \addnextpagesidebar[-1ex]{page3sidebar}

  \end{document}
